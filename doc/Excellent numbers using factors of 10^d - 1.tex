\documentclass[12pt,letterpaper]{article}

\usepackage{url}

\begin{document}

\title{Finding d-digit excellent numbers using factors of $(10^(d/2) - 1)$}

\author{A. Sinan Unur}

\date{January 17, 2016}

\maketitle

This is a re-write and expansion of comment on HN by a user called Someone. See \url{https://news.ycombinator.com/item?id=10909909} and \url{https://news.ycombinator.com/item?id=10915461}.

Suppose, for integers $a$, $b$, we have

\[ b^2 - a^2 = aK + b \]

where $K\equiv 10^(d/2)$ and $d\equiv\lceil\mathrm{log}_{10}a\rceil=\lceil\mathrm{log}_{10}b\rceil$.

Then:
%
\begin{eqnarray}
%
b^2 - b & = & a^2 + K \\
%
(b - \frac{1}{2})^2 - (\frac{1}{2})^2 & = & (a + \frac{K}{2})^2 - (\frac{K}{2})^2 \\
%
(2b - 1)^2 - 1 & = & (2a + K)^2 - K^2 \\
%
(2a + K)^2 - (2b - 1)^2 & = & K^2 - 1
%
\end{eqnarray}

Now, this equation is of the form $x^2 - y^2 = c$. Write $x^2 - y^2$ as $(x - y)(x + y)$:

\begin{equation}
%
\left[2(a - b) + (K + 1)\right] \left[2(a + b) + (K + 1)\right] = K^2 - 1 \label{eqn:fact1}
%
\end{equation}

Let $P$ be the set of factors of $(K^2 - 1)$. For every element $ = (p_1,p_2)\in S$, we have equations:

\begin{eqnarray}
%
\left[2(a - b) + (K + 1)\right] & = & p_1 \label{eqn:fact2}\\
%
\left[2(a + b) + (K + 1)\right] & = & q_1 \label{eqn:fact3}
%
\end{eqnarray}

For example, to find 38 digit excellent numbers, we need to generate the integer factors of $999,999,999,999,999,999$, then solve equations (\ref{eqn:fact2}) and (\ref{eqn:fact3}) for each pair of them.

This can be done off-line, and a look up table can be supplied to the program.

Computationally, because we have $K^2$ on the right hand side, we can get by using 128 bit arithmetic for numbers with up to 38 digits. Beyond that, we need 256 bit arithmetic.

Rearranging, we end up with the following sets of equations:

\begin{eqnarray}
%
a + b & = & \frac{q - (K + 1)}{2} \\
%
b - a & = & \frac{(K + 1) - p}{2}
%
\end{eqnarray}

Now, this is easy to solve by substitution:

\begin{equation}
    b = \frac{(K + 1) - p}{2} + a
\end{equation}

Therefore,

\begin{equation}
    a = \frac{(p + q) - 2(K + 1)}{4}
\end{equation}

and

\begin{equation}
    b = \frac{q - p}{4}
\end{equation}

\end{document}
